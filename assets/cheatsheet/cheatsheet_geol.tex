% geol CLI commands cheatsheet
% Version: __GEOL_VERSION__ | Commit: __GEOL_COMMIT__
\documentclass[a4paper,10pt]{article}
\usepackage[utf8]{inputenc}
\usepackage[T1]{fontenc}
\usepackage[english]{babel}
\usepackage{geometry}
\usepackage{multicol}
\usepackage{xcolor}
\usepackage{tcolorbox}
\geometry{margin=1.5cm, landscape, top=0cm, bottom=1cm, left=0.5cm, right=0.5cm}
\setlength{\columnsep}{1cm}
\usepackage{hyperref}
\hypersetup{
  pdftitle={opt-nc/geol CLI commands cheatsheet},
  pdfauthor={Adrien SALES, x.com/rastadidi},
  pdfsubject={Cheatsheet for geol CLI usage},
  pdfkeywords={geol, CLI, cheatsheet, EOL, endoflife.date, tips, cli, DEVOPS, DevSecOps}
}
\usepackage{fancyhdr}
\pagestyle{fancy}
\fancyhf{}
\fancyfoot[C]{\color{geoldark}\small Version: \_\_GEOL\_VERSION\_\_ \quad Commit: \_\_GEOL\_COMMIT\_\_ \quad Branch: \_\_GEOL\_BRANCH\_\_}
% Pictograms
\usepackage{fontawesome5}
\usepackage{qrcode}
\usepackage{eso-pic}
% Color definitions
\definecolor{geolblue}{HTML}{2563EB}
\definecolor{geolgreen}{HTML}{059669}
\definecolor{geolgray}{HTML}{F3F4F6}
\definecolor{geoldark}{HTML}{1E293B}
\definecolor{geolorange}{HTML}{F59E42}

% --- BEGIN PREAMBLE CODE ---
% STEP 1: Copy all of this section into your document's preamble.

% 1. Define the exact colors from your SVG logo
\definecolor{logoGreen}{RGB}{0,255,0}
\definecolor{logoOrange}{RGB}{255,165,0}
\definecolor{logoRed}{RGB}{255,0,0}

% 2. Create a reusable command for your logo
% Usage: \geolLogo[scale=1]
\newcommand{\geolLogo}[1][]{
    \begin{tikzpicture}[#1]
        % Common styles from your SVG
        \tikzstyle{logo_line} = [
            line width=10,      % Equivalent to stroke-width="20"
            line cap=round,     % Equivalent to stroke-linecap="round"
            line join=round     % Equivalent to stroke-linejoin="round"
        ]

        % Drawing commands, translated from SVG coordinates.
        % We use a 'yshift' and 'yscale' to flip the coordinate system
        % to match the SVG's top-left origin.
        \begin{scope}[yshift=-3.125pt, yscale=-1]
            % Green Paths
            \draw[logo_line, logoGreen] (50,50) -- (150,50);
            \draw[logo_line, logoGreen] (50,50) -- (100,100);
            \draw[logo_line, logoGreen] (150,50) -- (100,100);

            % Orange Paths
            \draw[logo_line, logoOrange] (100,100) -- (150,150);
            \draw[logo_line, logoOrange] (50,150) -- (100,100);

            % Red Path
            \draw[logo_line, logoRed] (50,150) -- (150,150);
        \end{scope}
    \end{tikzpicture}%
}

% --- END PREAMBLE CODE ---
% Title style
\usepackage{sectsty}
\sectionfont{\color{geolblue}}
% Command style
\newcommand{\cmd}[1]{\textcolor{geolgreen}{\texttt{#1}}}
% Box style
\newtcolorbox{geolbox}{colback=geolgray!80!white, colframe=geolblue, boxrule=0.8pt, arc=3pt, left=4pt, right=4pt, top=2pt, bottom=2pt}
% Compact itemize for geolbox
\newenvironment{geolitemize}{%
  \begin{itemize}\setlength{\itemsep}{0.1em}\setlength{\parskip}{0pt}\setlength{\topsep}{0pt}
}{\end{itemize}}
% Title styled as a terminal command
\usepackage{tikz}
\newtcolorbox{terminaltitle}{colback=black!95!white, colframe=geolgreen, boxrule=1pt, arc=3pt, left=0pt, right=0pt, top=0pt, bottom=0pt}
\renewcommand{\maketitle}{%
    \begin{center}
        \begin{terminaltitle}
            \centering
            \geolLogo[scale=0.02, baseline=-20pt] \\
            {\textcolor{logoGreen}{\texttt{geol}}} \\
            {\color{white}\href{https://github.com/opt-nc/geol}{opt-nc/geol}} \\
            {\color{white}\large \textit{"\textcolor{logoGreen}{\texttt{geol}}, because software End of Life management is too serious to be boring"}}
        \end{terminaltitle}
    \end{center}
}
\author{\href{https://github.com/opt-nc/geol}{opt-nc/geol}}


\begin{document}

\AddToShipoutPictureFG*{%
\begin{tikzpicture}[remember picture, overlay]
    \draw[logoGreen, line width=10pt, line cap=butt]
        ([yshift=-0.333\paperheight]current page.north west) --
        (current page.north west) --
        (current page.north east) --
        ([yshift=-0.333\paperheight]current page.north east);
    \draw[logoOrange, line width=10pt, line cap=butt]
        ([yshift=-0.333\paperheight]current page.north west) --
        ([yshift=-0.666\paperheight]current page.north west);
    \draw[logoOrange, line width=10pt, line cap=butt]
        ([yshift=-0.333\paperheight]current page.north east) --
        ([yshift=-0.666\paperheight]current page.north east);
    \draw[logoRed, line width=10pt, line cap=butt]
        ([yshift=-0.666\paperheight]current page.north west) --
        (current page.south west) --
        (current page.south east) --
        ([yshift=-0.666\paperheight]current page.north east);
\end{tikzpicture}}



\maketitle
\thispagestyle{fancy}
\renewcommand{\thefootnote}{\fnsymbol{footnote}}
\footnotetext[1]{Proudly brought to you with \faHeart \ \& \texttt{XeLaTeX}. | geol - Version: __GEOL_VERSION__ | Commit: __GEOL_COMMIT__ | Branch: __GEOL_BRANCH__}


% (presentation moved to the title)



\begin{multicols}{2}

\section*{\faTerminal\ Get It}

\begin{geolbox}
\begin{geolitemize}
  \item \cmd{brew install -{}-cask opt-nc/homebrew-tap/geol}
  \item \cmd{go install github.com/opt-nc/geol@latest}
  \item \cmd{geol help}
  \item \cmd{geol about}
\end{geolitemize}
\end{geolbox}


\section*{\faTags\ products, tags \& categories}

\begin{geolbox}
\begin{geolitemize}
  \item \cmd{geol list products} All products keys
  \item \cmd{geol list products -t} All products keys and aliases in a tree view
  \item \cmd{geol list categories} All categories keys
  \item \cmd{geol list tags} All tags keys
  \item \cmd{geol tag database} Show all products tagged as \cmd{database} 
  \item \cmd{geol list products -t | grep sql}: Show products that have \cmd{sql} in their name
\end{geolitemize}
\end{geolbox}

\section*{\faShareSquare\ Reporting}
\begin{geolbox}
\begin{geolitemize}
  \item \cmd{geol product extended temurin quarkus psql -n15 > geol-report.md} to export a report in \cmd{markdown} format 
  \item \cmd{pandoc geol-report.md -o geol-report.pdf --pdf-engine=xelatex} to convert the \cmd{markdown} report to PDF
  \item \cmd{pandoc geol-report.md -f markdown -t asciidoc -o geol-report.adoc} to convert the markdown report to \cmd{asciidoc}
  \item \cmd{asciidoctor -a toc -a toclevels=4 geol-report.adoc} to convert the \cmd{asciidoc} report to HTML
  \item \cmd{asciidoctor-pdf -a toc -a toclevels=4 geol-report.adoc} to convert the \cmd{asciidoc} report to PDF
\end{geolitemize}
\end{geolbox}


\section*{\faSearch\ Stack survey}
\begin{geolbox}
\begin{geolitemize}
  \item \cmd{geol check init}: Create a valid \cmd{.geol.yaml} check file
  \item \cmd{geol check}: Run the check based on default \cmd{.geol.yaml} file
  \item \cmd{geol check -s}: Run the check but exists with code 1 if any product is EOL
\end{geolitemize}
\end{geolbox}



\section*{\faBoxOpen\ Products}
\begin{geolbox}
\begin{geolitemize}
  \item \cmd{geol product describe psql} : Describe PostgreSQL product
  \item \cmd{geol product describe psql > psql.md} : Get PostgreSQL description in a markdown file
  \item \cmd{geol product extended psql} : Get PostgreSQL versions and EOL dates
  \item \cmd{geol product extended psql -n10} : Get PostgreSQL versions and 10 latest EOL dates
  \item \cmd{geol product extended psql > psql.md} : Get PostgreSQL versions and EOL dates in a valid markdown
  \item \cmd{geol product extended psql quarkus temurin maven > java-stack.md} : Get a collection of products in a single markdown
\end{geolitemize}
\end{geolbox}

\section*{\faBookmark\ Related contents}
\begin{geolbox}
\begin{minipage}[c]{0.7\linewidth}
\begin{geolitemize}
  \item \href{https://bit.ly/48sUBdf}{⏳geol, the cli to efficiently manage EOLs like a boss} : bit.ly/48sUBdf
  \item \href{https://bit.ly/3XWu0QI}{Dedicated Youtube playlist} : bit.ly/3XWu0QI
\end{geolitemize}
\end{minipage}%
\begin{minipage}[c]{0.3\linewidth}
\centering
\qrcode[height=1.5cm]{bit.ly/4irKyJX}
\end{minipage}
\end{geolbox}

\end{multicols}
\end{document}
